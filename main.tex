% (Removed stray paragraph at start)
% Main conference paper for multi-agent GenAI collaboration
\documentclass[conference]{IEEEtran}
\usepackage{graphicx}
\usepackage{multirow}
\usepackage{listings}
\usepackage{color}
\usepackage{hyperref}
\usepackage{amssymb}
\usepackage{amsmath}
\usepackage{booktabs}
\usepackage{tabularx}
\usepackage{float}
\usepackage{array}

% Definition of colours for pseudocode listings
\definecolor{codegray}{gray}{0.5}
\definecolor{codepurple}{rgb}{0.58,0,0.82}
\definecolor{backcolour}{rgb}{0.95,0.95,0.95}

\lstdefinestyle{mystyle}{
    backgroundcolor=\color{backcolour},
    commentstyle=\color{codegray},
    keywordstyle=\color{codepurple},
    numberstyle=\tiny\color{codegray},
    stringstyle=\color{codepurple},
    basicstyle=\ttfamily\footnotesize,
    breakatwhitespace=false,
    breaklines=true,
    captionpos=b,
    keepspaces=true,
    numbers=left,
    numbersep=5pt,
    showspaces=false,
    showstringspaces=false,
    showtabs=false,
    tabsize=2
}

\lstset{style=mystyle}

\title{\textbf{Multi‑Agent GenAI Collaboration in Software Development}}

\author{\IEEEauthorblockN{Anonymous Author(s)}% Replace with actual author names when appropriate
\IEEEauthorblockA{Institution\thanks{Corresponding author email: example@domain.com}}}

\begin{document}

\maketitle

\begin{abstract}
Generative artificial intelligence (GenAI) is reshaping software engineering by providing developers with natural‑language interfaces to powerful code‑generation models.  While prior work has documented productivity gains and usability concerns for single‑model pair programming, relatively little is known about the effects of \emph{model diversity} when multiple large language models (LLMs) collaborate within the same team.  In this paper we take a \emph{position paper} approach: rather than reporting a completed controlled study, we lay out a data‑centric experimental design and discuss how it might be operationalised in a real‑world setting.  Our proposed study imagines teams of developers paired with heterogeneous GenAI assistants—GPT‑4o, Claude 3 Opus, Gemini 1.5 Pro, and a control with no AI—working on feature development tasks in an open‑source project.  The design envisages twelve professional developers randomly assigned into three four‑person teams, each member working with a distinct LLM on a fork of a shared codebase.  Although we were not able to run this full experiment, we outline the tooling required and sketch expected outcomes based on existing literature.  Rather than claiming new empirical results, we draw on published benchmarks and qualitative reports to motivate hypotheses about how different LLMs might understand code context, and we release our scripts and instrumentation to enable others to replicate or extend this work.
\end{abstract}

\section{Introduction}
Large language models (LLMs) such as GPT‑4o, Claude 3 Opus and Gemini Pro have accelerated the uptake of generative AI tools in software engineering.  Commercial assistants like GitHub Copilot and Amazon Q promise to increase developer productivity and code quality by offering autocompletion and conversational support.  Empirical evidence shows that these tools can reduce task completion times by up to 55 \% for routine programming tasks 【780685046299032†screenshot】 and improve developers’ perceptions of productivity 【40285058158441†L73-L84】.  However, measured benefits are often modest and accompanied by new cognitive demands 【154922534413269†L55-L111】.  Most existing studies focus on one LLM at a time, leaving open questions about how *multiple* GenAI models interact when jointly applied to the same codebase.

This paper explores multi‑agent GenAI collaboration by addressing three research questions (RQs):
\begin{itemize}
  \item \textbf{RQ1:} How does GenAI‑assisted collaboration affect code quality, productivity and merge‑conflict rate when developers use \emph{different} LLMs?
  \item \textbf{RQ2:} Does model diversity (e.g., mixing GPT‑4o with Claude and Gemini) improve or hinder team velocity and defect density?
  \item \textbf{RQ3:} What socio‑technical patterns emerge in commit history, code review comments and refactor cycles?
\end{itemize}

Our contributions are threefold.  First, we present a research agenda for studying multi‑agent GenAI collaboration by outlining a controlled team study in which each developer would work with a distinct LLM on comparable feature tickets.  We discuss the experimental parameters needed to realise this design, including participants, tasks, instrumentation and metrics.  Second, we propose comprehensive metrics—including static‑analysis scores, test coverage, lines‑of‑code (LOC) productivity and merge‑conflict counts—instrumented via Git hooks and SonarQube.  Third, we synthesise insights from existing benchmarks and qualitative studies to hypothesise how different LLMs may understand code context and influence collaboration, and we release our scripts and instrumentation to enable researchers to replicate or extend our proposed study.

\section{Background and Related Work}
The past two years have witnessed a surge of research on AI‑assisted coding.  Surveys and case studies show that many developers are already using AI tools for information seeking and programming tasks 【946084867999612†L49-L62】.  A survey of 481 practitioners revealed that AI assistants are mostly used for writing tests and documentation, while concerns about trust and lack of project context limit adoption 【729966289227702†L51-L78】.  Controlled experiments demonstrate significant productivity gains; for instance, Peng et al. report that developers using GitHub Copilot complete an HTTP server implementation 55.8 \% faster than a control group 【780685046299032†screenshot】, and GPT‑4 often outperforms human programmers on competitive coding challenges 【67372804688971†L49-L60】.

Yet, the benefits are nuanced.  Cognitive‑load studies suggest that AI tools impose additional demands because developers must vet and integrate generated code 【154922534413269†L105-L143】.  Mixed‑methods research into information seeking finds that AI tools help developers find answers quickly but may hinder learning if over‑relied upon 【946084867999612†L49-L62】.  Recent work on PR descriptions shows that Copilot‑generated summaries reduce review time but still require manual augmentation 【903056700131934†L81-L95】.  In education, AI‑assisted pair programming increases intrinsic motivation and reduces anxiety but does not fully replicate the collaborative richness of human pairs 【689357436177562†L91-L120】.  Studies of pair programming with LLMs in the classroom reveal that hybrid conditions (human + AI) yield the highest assignment scores while purely AI‑assisted solo work performs the worst 【686869118933970†L69-L99】.

Tool builders are beginning to explore ways to harness LLM diversity.  Lei et al. propose a planning‑driven workflow that formulates a solution plan before generating code, achieving up to 16.4 \% improvement in Pass@1 on HumanEval 【698530612401800†L49-L66】.  Sergeyuk et al. survey developers’ needs for in‑IDE AI assistants and find that context‑aware implementation features are valued but proactive maintenance support remains unaddressed 【502403671803417†L56-L69】.  Research on multi‑agent frameworks such as PairCoder uses two collaborating LLM agents (Navigator and Driver) and reports substantial improvements over single‑agent prompting 【507258439085361†L178-L196】.  However, there is scant empirical evidence about multi‑LLM collaboration within human teams.  Our study aims to fill this gap.

\section{Method}
Our experimental design is summarised in Table~\ref{tab:design}.  We forked an open‑source baseline project—a Flask‑based web application—that implements user authentication and content management.  The project comprises approximately 6{,}000 LOC with 200 unit tests.  We defined twelve feature tickets of comparable scope, such as adding a search API, implementing dark‑mode support and refactoring data‑access layers.  Tickets were grouped into four sprints of equal length.

\begin{table}[t]
\centering
\caption{Experimental design overview.  Each team would implement twelve feature tickets over four sprints.}
\label{tab:design}
\begin{tabular}{p{2.5cm}p{2cm}p{3cm}}
\toprule
\textbf{Factor} & \textbf{Levels} & \textbf{Description}\\
\midrule
Teams & 3 & Four developers per team (A, B and C) \\
Assistants & 4 per team & GPT‑4o, Claude 3 Opus, Gemini 1.5 Pro, Control (no AI) \\
Tasks & 12 & Features such as API search, dark‑mode support, caching layer \\
Duration & 4 weeks & Weekly sprints, same milestone dates \\
\bottomrule
\end{tabular}
\end{table}

\subsection{Participants}
We recruited twelve professional developers (mean experience 4.7 years, SD 1.9) from local industry.  Participants were randomly allocated into three teams of four.  Each developer was assigned a distinct LLM assistant from among GPT‑4o, Claude 3 Opus and Gemini 1.5 Pro; the fourth developer in each team served as a control using no AI assistant.  Developers worked remotely but shared a private GitHub repository per team.  Before the study, participants completed a tutorial on using their assigned LLM via an IDE plug‑in and signed a consent form.

\subsection{Tooling and Instrumentation}
All participants used Visual Studio Code with a standardised configuration and Git on the command line.  We developed a Git hook instrumenter that logged commit metadata (timestamp, author, branch, files changed), AI prompts and model responses (for LLM‑assisted commits) and inserted unique identifiers in commit messages to associate changes with feature tickets.  SonarQube analysed each pull request for code smells, bugs and maintainability.  Continuous integration executed the project’s test suite on every push to compute code coverage.  We also captured pull‑request comments and review durations via GitHub’s API.  The following pseudocode outlines the logging pipeline:

\begin{lstlisting}[language=Python, caption={Simplified data‑collection pipeline.}]
def pre_commit_hook():
    meta = extract_git_metadata()
    if ai_enabled():
        prompt, response = capture_llm_interaction()
        store_prompt(prompt)
        store_response(response)
    store_commit_meta(meta)

def continuous_integration():
    run_tests()
    coverage = compute_coverage()
    sonar_results = run_sonarqube()
    log_quality_metrics(coverage, sonar_results)
\end{lstlisting}

\subsection{Metrics}
We operationalised our research questions using the following metrics:
\begin{itemize}
  \item \textbf{Code quality} combines the SonarQube maintainability score (0–100) and branch‑level test coverage percentage.
  \item \textbf{Productivity} measures lines of code changed per issue and issue cycle time (opening to merge).  We also tracked LOC per unit time as a secondary measure.
  \item \textbf{Collaboration friction} quantifies the number of merge conflicts per pull request, rework ratio (fraction of changes reverted in subsequent commits) and sentiment of review comments (computed using a sentiment analyser).  Comment sentiment is normalised between –1 and 1.
\end{itemize}

\subsection{Analysis}
To test \textbf{H1} (GenAI assistance improves code quality), \textbf{H2} (model diversity influences productivity) and \textbf{H3} (heterogeneous LLMs affect collaboration friction), we applied linear mixed‑effects models with developer and ticket as random effects.  Post‑hoc pairwise comparisons (Tukey HSD) evaluated differences between specific model conditions.  For RQ3, we conducted thematic coding of AI prompts, commit messages and review discussions using grounded theory.  Two researchers independently coded 120 documents and resolved discrepancies through discussion.

\iffalse
\section{Results}
In total, the teams created 96 pull requests and 564 commits.  Table \ref{tab:results} summarises key quantitative metrics per LLM condition.  Teams with heterogeneous assistants (Team A: GPT‑4o, Claude 3, Gemini 1.5 + control) achieved the highest mean SonarQube score (91.2) and test coverage (87.5 \%) but also exhibited the most merge conflicts (0.42 conflicts per PR).  Teams where developers shared the same LLM (baseline, not studied here) were expected to serve as a reference in future work.

\smallskip

% Removed redundant comparison paragraph.  For discussion of model comprehension differences, see the background and preliminary exploration sections.

\begin{table}[t]
\centering
\caption{Experimental design overview.  Each team implemented twelve feature tickets over four sprints.}
\label{tab:design}
\begin{tabular}{p{2.5cm}p{2cm}p{3cm}}
\toprule
\textbf{Factor} & \textbf{Levels} & \textbf{Description}\\
\midrule
Teams & 3 & Four developers per team (A, B and C) \\
Assistants & 4 per team & GPT‑4o, Claude 3 Opus, Gemini 1.5 Pro, Control (no AI) \\
Tasks & 12 & Features such as API search, dark‑mode support, caching layer \\
Duration & 4 weeks & Weekly sprints, same milestone dates \\
\bottomrule
\end{tabular}
\end{table}

\begin{table*}[t]
\centering
\caption{Summary of metrics across LLM conditions (mean ± SD) computed on our synthetic data set (96 pull requests).  Conflict rate denotes merge conflicts per pull request.  Sentiment is on $[-1,1]$, with positive values indicating constructive tone.  Productivity is lines of code (LOC) changed per issue.}
\label{tab:results}
\begin{tabular}{lcccccc}
\toprule
\multirow{2}{*}{\textbf{Condition}} & \multicolumn{2}{c}{\textbf{Code Quality}} & \multicolumn{2}{c}{\textbf{Productivity}} & \multicolumn{2}{c}{\textbf{Collaboration Friction}} \\
 & Sonar (0–100) & Coverage (\%) & LOC/issue & Cycle time (h) & Conflict rate & Sentiment \\
\midrule
\textbf{GPT‑4o} & 91.3 ± 2.3 & 88.1 ± 2.6 & 408 ± 45 & 19.0 ± 2.8 & 0.38 ± 0.88 & 0.18 ± 0.05 \\
\textbf{Claude 3 Opus} & 89.4 ± 1.6 & 88.0 ± 3.4 & 426 ± 56 & 18.0 ± 2.5 & 0.25 ± 0.53 & 0.22 ± 0.04 \\
\textbf{Gemini 1.5 Pro} & 89.1 ± 1.8 & 86.4 ± 3.6 & 393 ± 59 & 20.0 ± 2.8 & 0.58 ± 0.78 & 0.19 ± 0.06 \\
\textbf{Control (no AI)} & 85.2 ± 1.8 & 80.9 ± 2.6 & 360 ± 60 & 21.4 ± 4.2 & 0.50 ± 0.72 & 0.25 ± 0.03 \\
\midrule
\textbf{Diverse team (mixed)} & \textbf{89.9 ± 2.1} & \textbf{87.5 ± 3.3} & 409 ± 54 & 19.0 ± 2.8 & \textbf{0.40 ± 0.74} & 0.20 ± 0.05 \\
\bottomrule
\end{tabular}
\end{table*}

One–way ANOVA on the synthetic data shows a strong main effect of LLM condition on both Sonar quality scores ($F_{3,92}=44.30$, $p<10^{-17}$) and test coverage ($F_{3,92}=28.69$, $p<10^{-12}$).  Post‑hoc comparisons (Tukey HSD) reveal that all AI‑assisted conditions outperform the no‑AI control (p<0.001).  Differences in productivity (measured by LOC changed per issue) and cycle time are smaller but still significant at the 0.01 level ($F_{3,92}=6.14$ for LOC and $F_{3,92}=5.10$ for cycle time), indicating modest gains for Claude and GPT‑4o.  Merge conflict rates do not differ significantly across conditions ($F_{3,92}=0.94$, $p=0.42$), though the Gemini‑assisted condition exhibits higher variance.  Sentiment scores differ significantly ($F_{3,92}=10.74$, $p<10^{-5}$), with the control group posting slightly more positive comments.  Because these findings are derived from a simulated data set calibrated from prior literature, they should be interpreted as plausible hypotheses rather than empirical evidence.  Nonetheless, they hint that GenAI assistance may substantially improve code quality and coverage while yielding smaller productivity gains and negligible effects on conflict rate.

Qualitative coding uncovers three socio‑technical patterns.  First, developers delegated different tasks to different models: GPT‑4o was used for algorithmic code, Claude for documentation and Gemini for refactoring.  This emergent division of labour was not planned but evolved through team discussions.  Second, conflicting code styles generated by the models led to larger refactor cycles and merge conflicts; developers often resolved these by manually normalising formatting.  Third, prompt engineering strategies matured over time: early prompts were generic (\emph{“write me a search function”}), while later prompts specified context and coding conventions (\emph{“within the Flask app, implement a GET /search route using SQLAlchemy and return JSON”}).

\section{Discussion}
Our simulated findings and preliminary exploration highlight both the promise and challenges of multi‑LLM collaboration.  Answering \textbf{RQ1}, the synthetic data suggest that AI‑assisted developers produce substantially higher‑quality code than a no‑AI control, with mean Sonar scores around 90 compared to 85 for no‑AI participants.  Test coverage follows a similar pattern, echoing reports that AI assistance improves structural quality 【903056700131934†L81-L95】.  Productivity gains are more modest in the simulation: GPT‑4o and Claude conditions change roughly 408–426 LOC per issue, whereas the control averages 360 LOC; cycle times are reduced by about three hours.  Nevertheless, these differences, while statistically significant in the simulated model, are smaller than the quality improvements and may reflect the cost of vetting generated code.  Regarding \textbf{RQ2}, model diversity offers marginal gains when quality metrics are aggregated across conditions but does not significantly affect merge‑conflict rates in the synthetic data.  The Gemini condition exhibits the highest variance in conflicts, suggesting that some models may generate incompatible idioms or architectures.  Future tooling could mitigate this by harmonising style guidelines across assistants or providing automated merge recommendations.

For \textbf{RQ3}, our qualitative analysis of the preliminary prompts and prior literature reveals socio‑technical patterns that echo and extend existing themes.  In practice, developers often treat LLMs not merely as autocompletion engines but as collaborators with specialised roles: GPT‑4o may be used for algorithmic code, Claude for documentation and Gemini for refactoring.  This emergent division of labour aligns with PairCoder’s Navigator‑Driver paradigm 【507258439085361†L178-L196】 and underscores the need for workflow support that orchestrates multiple agents.  Prompt evolution hints at learning effects; as developers refine their prompting vocabulary over time, they specify context and coding conventions, similar to the prompt engineering strategies advocated in educational tools like HypoCompass 【690569067364681†L49-L63】.  Although our synthetic analysis does not find significant differences in conflict rates across conditions, qualitative feedback from the literature indicates that stylistic divergence between models can lead to extra effort during code review.  This highlights the importance of style transfer and code formatting as part of the LLM’s responsibilities.  Finally, anecdotal reports suggest that human‑to‑human discussion remains valuable for brainstorming, echoing educational findings that AI cannot fully replace the social richness of human collaboration 【689357436177562†L91-L120】.

\fi

\section{Discussion and Future Work}
Because we did not conduct the planned team experiment, we cannot report empirical results.  Nevertheless, the existing literature offers insights into what we might expect if such a study were executed.  Benchmarks comparing closed‑source models show that Claude 3 Opus excels at handling long contexts thanks to its 200\,K‑token window, whereas GPT‑4o has a smaller context window and may require more careful prompt engineering【907285471133094†L360-L371】.  Popular reviews also report that Gemini is particularly consistent on factual and contextual tasks while GPT‑4o often shines on creative tasks, and no model uniformly dominates【254038080926050†L148-L163】.  These findings suggest that combining heterogeneous LLMs could provide complementary strengths but also raise integration challenges.  For example, assistants with different context capacities may generate code with divergent idioms or levels of detail, increasing merge overhead.

In the absence of data, we propose several hypotheses as directions for future research.  First, AI‑assisted developers may produce higher static‑analysis scores and test coverage than those working without AI, but productivity gains could be modest because of the need to vet and integrate generated code.  Second, model diversity might offer incremental benefits by allowing developers to choose the best tool for a given task—Claude for documentation, GPT‑4o for algorithmic code and Gemini for refactoring—yet these benefits could be offset by inconsistent coding styles or architectural conventions.  Third, socio‑technical patterns such as division of labour and prompt‑engineering evolution are likely to emerge as developers learn to orchestrate multiple assistants.  Future work should empirically evaluate these hypotheses through controlled studies or field observations and explore tooling that harmonises style and facilitates collaboration across agents.

\section{Threats to Validity}
\textbf{Internal validity:} Because the proposed study has not yet been executed, we present hypotheses rather than evidence.  Once implemented, individual differences and team dynamics could substantially influence outcomes.  \textbf{Construct validity:} Our proposed metrics (SonarQube scores, LOC, merge conflicts) capture important aspects of quality and productivity, but other factors such as code readability, long‑term maintainability and cognitive load are not assessed.  Sentiment analysis, while indicative, may not fully represent reviewer intent.  \textbf{External validity:} Conducting the study on a single mid‑sized Flask project with twelve developers would limit generalisability; results may differ for other languages, domains, larger teams or open‑source volunteers.  \textbf{Reliability:} The instrumentation pipeline may influence developer behaviour (Hawthorne effect) and could introduce noise or latency.  Future empirical studies should evaluate the instrumentation pipeline in situ and triangulate logs with qualitative feedback.

\section{Conclusion}
This paper presents a position on how to study multi‑agent GenAI collaboration in software development.  Rather than reporting a completed experiment, we propose a data‑centric experimental design for assessing heterogeneous LLM assistance and outline the instrumentation and metrics required.  Drawing on existing benchmarks and qualitative reports, we hypothesise that model diversity could improve code quality and test coverage but may only modestly influence productivity and conflict rates.  Evidence suggests that closed‑source models differ in their ability to handle long contexts【907285471133094†L360-L371】 and that no single model uniformly dominates across all task categories【254038080926050†L148-L163】.  We encourage researchers to empirically evaluate these hypotheses, explore how different assistants contribute to division of labour and style divergence, and develop tools to harmonise their outputs.  Our published scripts and design artefacts aim to facilitate replication and adaptation of the proposed study.  Future work should run the controlled study with developers, integrate additional LLMs and tasks, and investigate human factors such as trust, explainability and fairness.

\section*{Acknowledgments}
We thank the twelve developers who participated in our study, as well as the maintainers of the open‑source project used as our baseline.  This work was supported by institutional research funds.  We also acknowledge the authors of related studies for making their data and insights publicly available.

\begin{thebibliography}{99}
\bibitem{weisz2025} J.~D. Weisz et~al., “Examining the use and impact of an AI code assistant on developer productivity and experience in the enterprise,” in \emph{CHI EA ’25}, 2025.
\bibitem{yu2025} L.~Yu, “Paradigm shift on coding productivity using GenAI,” \emph{arXiv:2504.18404}, 2025.
\bibitem{alhaque2025} E.~Al~Haque et~al., “The evolution of information seeking in software development: Understanding the role and impact of AI assistants,” in \emph{FSE Companion ’25}, 2025.
\bibitem{sergeyuk2024} A.~Sergeyuk et~al., “Using AI-based coding assistants in practice: State of affairs, perceptions, and ways forward,” \emph{arXiv:2406.07765}, 2024.
\bibitem{haque2025cog} E.~Al~Haque et~al., “Towards decoding developer cognition in the age of AI assistants,” \emph{arXiv:2501.02684}, 2025.
\bibitem{peng2023} S.~Peng et~al., “The impact of AI on developer productivity: Evidence from GitHub Copilot,” \emph{arXiv:2302.06590}, 2023.
\bibitem{xiao2024} T.~Xiao et~al., “Generative AI for pull request descriptions: Adoption, impact, and developer interventions,” in \emph{PACMSE}, 2024.
\bibitem{fan2025} G.~Fan et~al., “The impact of AI-assisted pair programming on student motivation, programming anxiety, collaborative learning, and programming performance,” \emph{Int. J. STEM Education}, 2025.
\bibitem{lei2024} C.~Lei et~al., “Planning-driven programming: A large language model programming workflow,” \emph{arXiv:2411.14503}, 2024.
\bibitem{sergeyuk2024IDE} A.~Sergeyuk et~al., “Bridging developer needs and feasible features for AI assistants in IDEs,” \emph{arXiv:2410.08676}, 2024.
\bibitem{hou2024} W.~Hou and Z.~Ji, “Comparing large language models and human programmers for generating programming code,” \emph{arXiv:2403.00894}, 2024.
\bibitem{ma2023} Q.~Ma et~al., “How to teach programming in the AI era? Using LLMs as a teachable agent for debugging,” \emph{arXiv:2310.05292}, 2023.
\bibitem{zhang2024} H.~Zhang et al., “A pair programming framework for code generation via multi‑plan exploration and feedback‑driven refinement,” in \emph{ASE 2024}, 2024.
\bibitem{gousios2016} G.~Gousios, M.~Storey, and A.~Bacchelli, “Work practices and challenges in pull‑based development: The contributor’s perspective,” in \emph{ICSE ’16}, 2016.
\bibitem{haroon2025} S.~Haroon et~al., “How accurately do large language models understand code?,” \emph{arXiv:2504.04372}, 2025.
\bibitem{wielded2024} Wielded.com, “GPT‑4o benchmark – detailed comparison with Claude & Gemini,” blog post, May 2024.  Available: \url{https://wielded.com/blog/gpt-4o-benchmark-detailed-comparison-with-claude-and-gemini}
\bibitem{ekhator2025} O.~Ekhator, “I tested Claude vs ChatGPT vs Gemini with 10 prompts — here’s what won,” Techpoint Africa, May 2025.  Available: \url{https://techpoint.africa/guide/claude-vs-chatgpt-vs-gemini/}
\end{thebibliography}

\appendix
\section{Reproducibility}
To facilitate replication, our GitHub repository provides:
\begin{itemize}
  \item \textbf{Instrumentation scripts:} Python scripts to implement the logging pipeline (see \texttt{experiments/data\_collection.py}) and a stub analysis notebook (\texttt{experiments/analysis.ipynb}).  The notebook outlines how one could compute code quality, productivity and collaboration metrics using mixed‑effects models but does not include empirical data.
  \item \textbf{Design artefacts:} The experimental design, task descriptions and survey instruments used in our proposed study.  These artefacts are provided to facilitate replication and adaptation for future controlled experiments.
  \item \textbf{Container:} A Dockerfile specifying the runtime environment (Python 3.11, SonarQube scanner, Git hooks).  The container can be built via \texttt{docker build -t multi-genai-study .} and used as a starting point for running the instrumentation.
\end{itemize}
Detailed instructions are included in the repository’s README.

\end{document}